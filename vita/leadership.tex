
\hypertarget{leadership}{%
\section{Leadership}\label{leadership}}

% CReST Lab
During my time at UNCC, I have tried to take on leadership roles in every aspect of my job. As a faculty member, I view one of my primary roles as one where I demonstrate for students what leadership is and how best to lead others. In that regard, I see several leadership roles that I have taken on in my time here.

The first is the formation of the Cyber Resiliency, Security, and Trust (CReST) Lab. I have modeled my research lab in the same way that the Systems and Internet Infrastructure (SIIS) Lab at Penn State was organized during my tenure as a graduate student in the lab. My advisor worked tirelessly, encouraging students to strive for the best, but also to help others. It became a family, with each member of the lab supporting each other. From simple things like reading a paper before a deadline and offering feedback to everyone showing up for major presentations as a show of solidarity and support. I have strived to teach my students the same ideas, and demonstrate for them that we are all working together not only to achieve our academic goals, but to support each other in every way posssible. My students have taken this idea and run with it, organizing social events, and supporting each other whenever possible, even as some of the first students begin to transition into new careers, they remain in contact with their fellow students, offering whatever support they can. My goal is to model for my students a leader that is willing to work hard and support them not just in their academic career, but even as they move on from UNCC.

% 49th Security division
In addition to the CReST Lab, I work closely with students as a faculty advisor for the 49th Security Divison. This is another area where my goal is to model for the students, those qualities that are vital to being a strong leader. I meet reguarly with the student officers of the club to discuss current matters, and offer advice whenever possible on how to achieve their goals. One area where this is starting to show is in the activity surrounding cyber security competitions. The students form teams to participate in competitions, and I work with them to prepare for the competitions both in the classroom and when practicing outside of class. Many of the students express hesitation becuase they have not participated in these competitions before and I work with other students to encourage everyone to participate, regardless of skill level. I find that this has increased involvement in the club activities by many students that otherwise may not have joined and participated.

% Shaping IT within CCI
As I have worked on building the CReST Lab and also supporting the 49th Security Division, I have worked closely with the Technology Solutions Office (TSO) on how research computing equipment is managed within the College. I have worked with several other faculty to interface with both TSO and University IT Services to identify policies and procedures that have delayed research progress. I have worked with these groups to identify a path forward that satisfies the tensions between IT security, University policy, and researchers that have ``non-traditional'' IT needs. Through these efforts, I feel that I have developed a working relationship with TSO and University IT Services that will enable the College to adapt to ever-changing research needs while maintaining a collaborative working environment that satisfies all parties.

% TaPP leadership
Finally, I have been actively involved in the International Workshop on the Theory and Practice of Provenance (TaPP) for several years. I started as an author, submitting my research to the workshop. I then became involved in the program committee for TaPP, becoming a vocal proponent of shaping TaPP as a venue for research in the area of systems provenance. Through my participation on the program committee, I was invited to serve as co-PC chair in 2019, where I was able to advocate more strongly for systems researchers looking for a venue to submit their work. Finally, this coming year (Summer 2020), I have been invited to serve as General Chair for Provenance Week, a series of co-located events that includes TaPP. I have also been invited to sit on the steering committee for TaPP where I will continue to play an active role in shaping the workshop into a venue for systems researchers to publish their work on data provenance.
