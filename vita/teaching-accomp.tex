\hypertarget{accomplishments}{%
\subsection{Major Accomplishments}\label{accomplishments}}

In the past two years, I have worked to transform the course ITIS 3246 (previously ITIS 3110/L) (\emph{IT Infrastructure and Security}) to take advantage of active learning techniques that I have observed and learned about here within the College of Computing and Informatics. The course was originally taught as a traditional lecture/lab style course, and based on feedback from students during my first semester, I converted the course to a flipped classroom. While the initial response and feedback from the students was not positive, I have found that students generally enjoy the course and several have returned in subsequent semesters to tell me how much they learned in the course, and how those skills are now helping them in their jobs and internships.

In the Spring 2018 semester, I taught a ``Topics in SIS'' course titled \emph{Competitive Cyber Defense}. The goal of the course is to help students prepare for competitions that expose them to a hostile environment. In this environment, they are acting as the defenders of an enterprise network that is under active attack. The goal of the competition is to defend the network while also carrying out traditional IT functions, such as installing new software, adding user accounts, and ensuring that services remain online. In Spring 2019, I expanded this course to include a graduate section, and also submitted two Curriculog proposals to make these courses permanent offerings.

Much like ITIS 3246, I have decided to teach \emph{Competitive Cyber Defense} as a flipped class. In Spring 2019, I explored using short lectures that aim to enhance the material studied outside of the classroom. These lectures forgo traditional slides, and instead I use the Socratic method to guide the lecture, with students providing many of the ideas. I have found this to be immensely successful in engaging the students in the discussion, and it is a practice that I plan to continue in all of my classes.