\clearpage
\hypertarget{teaching-statement}{%
\section{Teaching Statement}\label{teaching-statement}}

% Hands on
I am a firm believer that the best way to learn is by doing, and this a how I teach every class that I have taught while at UNC Charlotte. I have been fortunate that the classes I teach have lent themselves to this style of teaching, where students have an opportunity to apply the techniques and skills that I teach. My experiences have led to to further understand the value in having skilled practitioners in the room to assist students. I have also learned through interaction with my peers here in CCI that this is one type of ``active learning'', where I am helping the students not only with the skills being taught in my class, but also teaching them how to become lifelong learners, something that I value greatly in my own life, and am proud to help students on this path of a lifetime of learning.

% Passive consumption is not for the classroom
My education was a very traditional education, with many classes being taught as standard lectures, interspersed with lab assignments when the topic warranted some amount of hands on activity. While I was able to succeed in such an environment, I have first-hand experience with several of my peers as a student who struggled with this style of teaching, often expressing a distaste for lectures, and seeming to only enjoy the labs. This caused me to question if this style of teaching was really most effective, especially in a computer science curriculum where many of the concepts can be read about on-line and then applied in a ``safe environment''. This idea has permeated my own teaching style in such a way that I try to integrate as many hands-on experiences for students as possible, especially in the classroom, where myself and the TA can serve as guides.

When I first considered the flipped classroom approach, where lectures would be presented as videos that students would be expected to watch before class, I asked students what they thought of the idea. Initially students were vocal about how much they did not enjoy this style of teaching. I began to question them, trying to understand the underlying problem, knowing that my own experience of a traditional lecture-based education was not the most ideal method for learning computer science. What I discovered was that students felt that the ``lecture material'' was being outsourced to others, such as Coursera, edX, and other on-line MOOCs, instead of professors recording their own lecture material and then providing short in-class discussions about the lectures. This has stuck with me, and I have ensured that all of my lectures are recorded by me, and fit the topics being covered in the course, only providing links to other videos as additional resources, instead of being the primary resource.

% Socratic lecture style
When I first started recording lectures, it became abundantly clear to me that I still needed to find ways to encourage students to watch the videos before class, enabling them to come to class prepared to apply the concepts they had seen in the videos. I tried a variety of different techniques over the last few semesters with varying degrees of success. At first, I created quizzes that would touch on the topics in the videos, but I found that students could easily search for many of the answers, avoiding having to watch the videos. I found that students would only watch videos in preparation for exams, a technique which has not worked well for them, given the volume of information being covered. Next, I started including questions on the assignments instead of as separate quizzes. These questions required them to think more critically about the content, such that a simple Google search was no longer sufficient to answer the questions. This seemed to be effective for some, but many expressed a dislike for this style of question, feeling that they were often subjective. I have kept these questions on the assignments, but have added some additional discussion time at the beginning of each class to cover the subjective side of these topics.

At the beginning of each class, I know set aside roughly 30-45 minutes (or more if necessary) where I have a Socratic-style lecture/discussion about the topic being covered. I have no slides to present, and instead ask questions and let the students guide the discussion. I keep a list of important points that need covered, but let the students guide the discussion through pointed questions. I have found that students have responded positively to this style of discussion, saying that they feel that the discussion was allowed to cover the topics that students had questions about, and could move quickly through the topics that students demonstrated a firm understanding of by answering my initial questions quickly. The most satisfying experience during this style of discussion was witnessing normally quiet students speaking up to answer another student's question before I had a chance to answer.

My journey as a teacher has only just begun, but I feel that I have grown tremendously in my first two years as an educator. I have moved from a belief that traditional lectures are the correct style of teaching to understanding that there is not a single ``best'' teaching style, and instead, that my role as an educator is to provide students with the tools they need to become lifelong learners.