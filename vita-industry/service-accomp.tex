\hypertarget{service-accomplishments}{%
\subsection{Accomplishments}\label{service-accomplishments}}

During my career, I have been fortunate to serve on many program and organizing committees as part of my external service. I have been actively involved in several of the top computer and network security conferences both as a program committee member and as an organizing committee member. As part of being actively involved, I was invited this past year to serve as co-PC chair for the International Workshop on the Theory and Practice of Provenance (TaPP). This workshop is one that is directly relevant to my research, and in 2020, I will serve as the General Chair for Provenance Week, a series of co-located workshops all focused on data provenance.

Within the College, I have taken an active role in serving as a mentor for the 49th Security Division, the ethical hacking and computer security student organization. I have worked closely with the officers of the club to develop changes to their constitution to ensure that all students are welcome within the organization. I have also worked to help the students develop a testbed for the club that is sustainable. This work has resulted in consolidating their infrastructure and teaching them best practices in infrastructure management, concepts which they are first exposed to in some courses, but rarely without a hands-on component that addresses server maintenance and datacenter operations.

Additionally, this year I have started advising undergraduates within the College as part of the Undergraduate Advising Committee. Previously, student advising was handled by the advising center and then by the teaching faculty within the departments. I feel that this role has been valuable in helping to understand the challenges students face and will help shape proposed changes to the curriculum and policies that will be in the best interest of the students.

Finally, one of my goals in the past two years has been to unify the security faculty within the College both professionally and socially. The end result will be a research group that is strong and collaborative, which will help teach our students how to conduct collaborative research. As part of these efforts, I have initiated a ``security reading group'' and a ``security social hour''. Both are open to all that have an interest in security and aim to expose students to new ideas that they otherwise might not be exposed to. These are activities that I will continue to coordinate in the coming years, in order to enhance the collaboration between the faculty and students.
