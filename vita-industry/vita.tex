\documentclass[10pt]{article}

\usepackage{hyperref}
\usepackage{enumitem}
\usepackage[backend=biber,style=verbose,maxbibnames=99,maxcitenames=99]{biblatex}
\usepackage{comment}
\usepackage{fancyhdr}
\usepackage[margin=1in]{geometry}
\usepackage{totpages}
\usepackage{etoolbox}
\usepackage{framed}

\newenvironment{quotationb}%
{\begin{leftbar}\begin{quotation}}%
{\end{quotation}\end{leftbar}}

\newtoggle{REVIEW}
\togglefalse{REVIEW}
%\toggletrue{REVIEW}

\hypersetup{colorlinks,%
  citecolor=black,%
  filecolor=black,%
  linkcolor=black,%
  urlcolor=black}

%\textheight=9.0in
\raggedbottom
\raggedright
\setlength{\tabcolsep}{0in}
\hyphenpenalty=10000

\pagestyle{fancy}
\fancyhead[C]{Thomas Moyer}
\fancyhead[L,R]{}
\fancyfoot[C]{Curriculum Vitae}
\fancyfoot[L]{February 2021}
\fancyfoot[R]{\thepage}
\renewcommand{\headrulewidth}{0pt}
\renewcommand{\footrulewidth}{0.4pt}
\setlength{\parskip}{0.65em}

\providecommand{\tightlist}{%
  \setlength{\itemsep}{0.25em}}

\addbibresource{moyer.bib}

\begin{document}
\begin{tabular*}{6.5in}{l@{\extracolsep{\fill}}r}
  \multicolumn{2}{c}{\Large{\textsc{Curriculum Vitae}}}
  \vspace{1em}\\
  Thomas Moyer\\
  177 Bromley Village Drive, Unit 201 &  \href{mailto:tommoyer@gmail.com}{tommoyer@gmail.comu}\\
  Fort Mill, SC 29708 & (814) 933-8135
  \\
\end{tabular*}
\\

\vspace{0.1in}
\hypertarget{education}{%
\section{Education}\label{education}}

\renewcommand{\labelitemi}{}
\begin{itemize}
\tightlist
\item
  \begin{tabular*}{6in}{l@{\extracolsep{\fill}}r}
    \href{http://www.psu.edu}{\textbf{The Pennsylvania State University}} & University Park, PA \\
    PhD, Computer Science and Engineering, December 2011 & \\
    Advisor: \href{http://www.patrickmcdaniel.org}{Dr. Patrick D. McDaniel} & \\
    Dissertation Title: \em{Building Scalable Document Integrity Systems} & %\\
  \end{tabular*}

\item
  \begin{tabular*}{6in}{l@{\extracolsep{\fill}}r}
    \href{http://www.psu.edu}{\textbf{The Pennsylvania State University}} & University Park, PA \\
    MS, Computer Science and Engineering, 2009 & \\
    Advisor: \href{http://www.patrickmcdaniel.org}{Dr. Patrick D. McDaniel} & \\
    Thesis Title: \em{Scalable Web Content Attestations} & %\\
  \end{tabular*}

\item
  \begin{tabular*}{6in}{l@{\extracolsep{\fill}}r}
    \href{http://www.psu.edu}{\textbf{The Pennsylvania State University}} & University Park, PA \\
    B.S., Computer Engineering, 2006 &
  \end{tabular*}

\end{itemize}
\renewcommand{\labelitemi}{\textbullet}

\hypertarget{professional-experience}{%
\section{Professional Experience}\label{professional-experience}}

\begin{itemize}
\tightlist
\item \textbf{Assistant Professor}, August 2017-present\\
  University of North Carolina at Charlotte\\
  Department of Software and Information Systems\\
  Charlotte, NC\\
  \textit{
Conducted research on a number of related areas: 1) novel solutions for building resilient systems that ensured continued operation in the face of compromise; 2) secure software-defined networking to support resilient systems; 3) trustworthy endpoint agent designs. Developed architectures that aimed to be efficient and secure against advanced adversaries. Techniques explored included data provenance and graph analysis. Target systems included cloud, enterprise, and embedded systems. Also co-led a project funded by the Department of Energy that involved working with students to carry out penetration testing exercises against a clone of a manufacturing system, and reported the findings to the program sponsor.\\
\mbox{}\\
Developed and taught two classes. The first was titled IT Infrastructure and Security where students learned about the configuration and management of enterprise services (e.g., DNS, LDAP, Kerberos, network storage, email servers, and web servers). I restructured the class to introduce containers and virtualization into the curriculum and integrated each of the assignments culminating in a functioning email server for a small enterprise network. The second class was titled Competitive Cyber Defense, an advanced undergraduate and graduate class, where students examined techniques for building secure architectures for enterprise services. In this class students would develop a plan to secure an existing enterprise network, including system and network security policies.
\mbox{}\\
Served as a mentor for several PhD, MS, and undergraduate student projects, helping students develop technical and professional skills. Helped with issues surrounding career path selection, technical skill development, and professional skill development. Additionally, I established a research lab within the university focused on research into resilient systems. In this lab, students worked collaboratively on research projects under my supervision.
\mbox{}\\
Acted as faculty mentor for the 49th Security Division, a student organization that focused on cybersecurity and ethical hacking. As part of this role, I helped students prepare for cybersecurity competitions such as the National Collegiate Cyber Defense Competition and the Department of Energy Cyberforce competition.}

\item \textbf{Research Scientist}, September 2011 to August 2017\\
  \textbf{Secure Resilient Systems and Technology Group}, MIT Lincoln Laboratory, Lexington, MA\\
\textit{
Principal investigator and designer of the security architecture for a mobile satellite communications terminal. Created a novel security architecture that met the security and information assurance requirements of the funding agency. The architecture relied on Linux-based virtualization and trusted computing hardware (i.e. the Trusted Platform Module) to ensure the integrity and confidentiality of the data and code for the platform.\\
\mbox{}\\
Worked on several programs related to cloud computing security and data provenance, including a program to integrate data provenance into applications that would ensure the integrity of the data being used to make decisions. The data provenance system included an analytic framework that would alert end-users to issues related to the data being accessed. Developed a lightweight data provenance collection system that could be integrated into existing security architectures where information assurance accreditation was required.\\
\mbox{}\\
Served as a technical advisor to the Department of Defense within the office of the Assistant Secretary of Defense for Research and Engineering. Evaluated research programs and proposals and provided guidance on future directions for program investments.}

\item \textbf{Research Assistant}, May 2008 to September 2011\\
  The Pennsylvania State University\\
  University Park, PA\\
  Advisor: Dr. Patrick D. McDaniel\\
  \textit{Studied and developed systems to support integrity of web content protected by the Trusted Platform Module. Developed efficient solutions to certify the integrity of content, ensuring a minimal reduction in throughput while providing stronger security guarantees about the content being served. Explored solutions for secure storage architectures as part of a collaborative project. Developed solutions to demonstrate novel security solutions based on upcoming storage architectures.}

\item \textbf{Research Assistant}, September 2007 to May 2008\\
  The Pennsylvania State University
  University Park, PA\\
  Advisor: Dr. Patrick D. McDaniel\\
  Mentor: Dr. Subhabrata Sen\\
  \textit{ Worked on problems in configuration management. Assisted in developing/testing tool for creating router configurations.}

\item \textbf{Summer Research Intern AT\&T}, May 2007 to September 2007\\
  Internet and Networking Systems Research Center, AT\&T Labs Research,\\
  Florham Park, NJ\\
  Mentor: Dr. Subhabrata Sen\\
  \textit{Worked on problems in configuration management. Assisted in developing internal tool for creating configurations.}

\item \textbf{Instructor}, January 2007 to May 2007\\
  The Pennsylvania State University\\
  University Park, PA\\
  Department of Computer Science Engineering, Pennsylvania State University\\
  \textit{Taught Introduction to Algorithmic Processes (CMPSC 101). Instructed students in program design and creation using the MS Visual Basic programming language.}

\end{itemize}

\hypertarget{skills-and-experience}{%
\section{Selected Skills and Technology Experience}\label{skills-and-experience}}
\textit{Ordered by experience}
\begin{itemize}
\tightlist
\item \textbf{Programming languages:} Python, C, Bash, Java, C++, Ruby
\item \textbf{Linux system administration:} automation, management, deployment, kernel programming
\item \textbf{Security:} public-key cryptography, mandatory access control policies (SELinux and AppArmor), trusted hardware (TPM and Intel SGX), firewall administration and policy development
\item \textbf{Operating systems}: Linux, macOS, FreeBSD, OpenBSD
\item \textbf{Cloud technology:} Virtualization, Containers (LXC, LXD, Docker), Proxmox Virtual Environment, VMware ESXi and vSphere

\end{itemize}

%\hypertarget{licenses-and-certifications}{%
%\section{Licenses and Certifications}\label{licenses-and-certifications}}
%\addcontentsline{toc}{section}{licenses-and-certifications}

\iftoggle{REVIEW}{
  \hypertarget{career-highlights}{%
  \section{Career Highlights}\label{career-highlights}}
  
\begin{itemize}
\item Established the Cyber Resiliency Security and Trust (CReST) Lab and recruited two students into the PhD program from UNC Charlotte
\item Supervised students in undergraduate research experience that resulted in several students continuing their studies through either the MS program or the PhD program
\item Established the \emph{Competitive Cyber Defense} courses at the undergraduate and graduate levels (ITIS 4246/5246)
\item Actively engaged with the 49th Security Division to develop a plan for becoming competitive in cybersecurity competitions, and to increase their inclusion and diversity activities
\item Applied active learning techniques to \emph{IT Infrastructure and Security}, creating a flipped classroom environment (ITIS 3246)
\item Published several papers in top cyber security conferences (ACM CCS and ISOC NDSS) as well as papers in other venues
\item Deployed a scalable testbed infrastructure for both classes and research using open source software
\item Worked closely with University IT Services and the Technology Solutions Office to develop new policies and procedures for research equipment management
\item Worked with faculty in the College of Engineering to secure funding through the Department of Energy that will leverage the data provenance research I have been developing for the past several years
\item Served on organizing and program committees for several major conferences, leading to being named co-PC chair of the International Workshop on the Theory and Practice of Provenance in 2019 and General Chair of Provenance Week in 2020
\end{itemize}
}

\hypertarget{publications}{%
\section{Publications}\label{publications}}

\hypertarget{peer-reviewed-journal-publications}{%
\subsection{Peer Reviewed Journal
Publications}\label{peer-reviewed-journal-publications}}

\begin{enumerate}
\tightlist
\item \fullcite{bth+2017}
\item \fullcite{mbs+2012}
\item \fullcite{smj+2011}
\item \fullcite{bmm+2010}
\item \fullcite{emm+2009}
\end{enumerate}

\hypertarget{peer-reviewed-conference-publications}{%
\subsection{Peer Reviewed Conference
Publications}\label{peer-reviewed-conference-publications}}

\begin{enumerate}[resume]
\tightlist
\item \fullcite{aamk2019}
\item \fullcite{phm+2018}
\item \fullcite{hbm2018}
\item \fullcite{phg+2017}
\item \fullcite{bbd+2017}
\item \fullcite{scm+2016}
\item \fullcite{mcc+2016}
\item \fullcite{mg2016}
\item \fullcite{btb+2015}
\item \fullcite{mjm2012}
\item \fullcite{hrk+2010}
\item \fullcite{mbs+2009}
\item \fullcite{sms+2009}
\end{enumerate}

\hypertarget{peer-reviewed-extended-abstractsshort-papers}{%
\subsection{Peer Reviewed Extended Abstracts/Short
Papers}\label{peer-reviewed-extended-abstractsshort-papers}}

\begin{enumerate}[resume]
\tightlist
\item \fullcite{smm2018}
\item \fullcite{lhm+2017}
\item \fullcite{bbm2015}
\item \fullcite{smv+2010}
\end{enumerate}

\hypertarget{other-publications}{%
\subsection{Other Publications}\label{other-publications}}

\begin{enumerate}[resume]
\tightlist
\item \fullcite{bbd+2016}
\item \fullcite{NAS-149}
\item \fullcite{security10-summaries}
\item \fullcite{webapps10-summaries}
\item \fullcite{NAS-0127}
\item \fullcite{NAS-00114}
\item \fullcite{NAS-00103}
\item \fullcite{NAS-0098}
\item \fullcite{NAS-0095}
\end{enumerate}


\iftoggle{REVIEW}{
  \hypertarget{funding}{%
  \section{Extramural Funding}\label{funding}}
  
\setcounter{subsection}{5} % Jump to 6.6
\hypertarget{other-grants}{%
\subsection{Other Grants}\label{other-grants}}

\begin{enumerate}
\item \textit{Basic Ordering Agreement: Securing American Manufacturing (SAM)}, Lead PI: Aidan Browne, Co-PI: \textbf{Thomas Moyer}, Co-PI: Wesley Williams, Funding agency: Department of Energy (DoE), Amount: \$403,429.00, Program start: October 1, 2019, Program End: September 30, 2020, Purpose: Research
\item \textit{Basic Ordering Agreement: Securing American Manufacturing (SAM)}, Lead PI: Aidan Browne, Co-PI: \textbf{Thomas Moyer}, Co-PI: Stacey Watson, Co-PI: Wesley Williams, Funding agency: Department of Energy (DoE), Amount: \$660,840.00, Program start: May 15, 2018, Program End: September 30, 2019, Purpose: Research
\end{enumerate}

\hypertarget{unfunded-proposals}{%
\subsection{Unfunded Proposals}\label{unfunded-proposals}}

\begin{enumerate}
\item \textit{Retroactive Data and System Security Enforcement for Aggregated Photovoltaic Systems and Power Grid Resilience}, Lead PI: Weichao Wang, Co-PI: \textbf{Thomas Moyer}, Co-PI: Linquan Bai, Co-PI: Badrul Chowdhury, Co-PI: Meera Sridhar, Funding agency: Department of Energy (DoE), Amount: \$4,109,753.00, Purpose: Research
\item \textit{Provenance-driven Mission System Resilience}, Lead PI: Thomas Moyer, Funding agency: DOD Defense Advanced Research Projects Agency (DARPA), Amount: \$733,746.00, Purpose: Research
\item \textit{Resilient Programmable Logic Controllers}, Lead PI: Thomas Moyer, Funding agency: NSF I/UCRC: Center for Configuration Analytics and Automation, Amount: \$50,000.00, Purpose: Research
\end{enumerate}

}

\hypertarget{students}{%
\section{Student Supervision}\label{students}}

\hypertarget{phd-students}{%
\subsection{Doctoral Students Supervised}\label{phd-students}}

\begin{itemize}
\tightlist
\item \textbf{Maya Kapoor}, Dissertation co-advisor, Project title: \textit{Multi-layer Provenance Graph Analysis using Graph Representational Learning}, Degree: PhD, Completion date: \textit{expected Spring 2024}
\item \textbf{Ambarish Regmi}, Dissertation advisor, Project title: \textit{Trustworhty Endpoint Agents for Enterprise Systems}, Degree: PhD, Completion date: \textit{expected Spring 2022}
\item \textbf{Trevon Williams}, Dissertation advisor, Project title: \textit{Software-defined Networking for Resilience}, Degree: PhD, Completion date: \textit{expected Spring 2024}
\item \textbf{Enas Al Kawasmi}, Dissertation advisor, Project title: \textit{A Secure Decentralized Storage Platform for Data Provenance}, Degree: PhD, Completion date: \textit{expected Spring 2023}
\item \textbf{Abdullah Al Farooq}, Dissertation advisor, Project title: \textit{Enforcing Security Policies with Data Provenance to Enrich the Security of IoT/Smart Building System}, Degree: PhD, Completion date: Summer 2020, Current Position: Assistant Professor, Wentworth Institute of Technology, Boston, MA
\end{itemize}

\hypertarget{ms-students}{%
\subsection{Masters Students Supervised}\label{ms-students}}

\begin{itemize}
\tightlist
\item \textbf{Trevon Williams}, Thesis advisor, Project title: \textit{A Programmable Approach for a Resilient SDN Architecture}, Degree: MS Cybersecurity, Completion date: Fall 2019
\item \textbf{Mir Mehedi Pritom}, Academic advisor, Degree: MS IT, Completion date: Fall 2018
\item \textbf{Anibal J. Robles Perez}, Thesis advisor, Project title: \textit{Towards and Agent-based Approach to Simulating Humans Falling for Phishing Attacks}, Degree: MS IT, Completion date: Fall 2018
\end{itemize}

\hypertarget{bs-students}{%
\subsection{Bachelors Students Supervised}\label{bs-students}}

\begin{itemize}
\tightlist
\item \textbf{Cameron Pacileo}, Project supervisor, Project title: \textit{Rollback-aware Trustworthy Data Provenance}, Degree: BS, Completion date: \textit{expected Spring 2022}
\item \textbf{Bryce Kane}, Project supervisor, Project title: \textit{Building a Safe and Scalable Testbed for System Security Research}, Degree: BS, Completion date: \textit{expected Spring 2022}
\item \textbf{Zachary Taylor}, Project supervisor, Project title: \textit{Enhancing Trustworthy Whole-system Provenance Analysis with Network Events}, Degree: BS, Completion date: \textit{expected Fall 2019}
\item \textbf{Kevin Cardoso}, Project supervisor, Project title: \textit{Building a Safe and Scalable Testbed for System Security Research}, Degree: BS, Completion date: Spring 2019
\item \textbf{Joe Waller}, Project supervisor, Project title: \textit{Enhancing Trustworthy Data Provenance Systems with Network Event Tracking}, Degree: BS, Completion date: Spring 2019, Note: PhD student starting in Fall 2019
\item \textbf{Joeseph Logan}, Project supervisor, Project title: \textit{Rollback-aware Trustworthy Data Provenance}, Degree: BS, Completion date: Spring 2019
\item \textbf{Abdalla El-Ghannam}, Project supervisor, Project title: \textit{Exploring Resource-aware Data Provenance Collection in Embedded Devices}, Degree: BS, Completion date: Spring 2019
\item \textbf{Jessica Marquard}, Project supervisor, Project title: \textit{Building Resilient Microservices with Data Provenance}, Degree: BS, Completion date: Summer 2018
\item \textbf{Kailey Wolfe}, REU faculty mentor, Project title: \textit{Exploration of Graph Databases for Secure Data Provenance}, Degree: BS, Note: Summer 2018 REU Student
\item \textbf{Karena Huang}, REU faculty mentor, Project title: \textit{Exploration of Graph Databases for Secure Data Provenance}, Degree: BS, Note: Summer 2018 REU Student
\item \textbf{Kripa George}, REU faculty mentor, Project title: \textit{Identifying Conflicts in Provenance Graphs for IoT/Smart Buildings}, Degree: BS, Note: Summer 2019 REU Student
\end{itemize}


\hypertarget{teaching}{%
\section{Teaching}\label{teaching}}

\iftoggle{REVIEW}{
  \hypertarget{accomplishments}{%
\subsection{Major Accomplishments}\label{accomplishments}}

In the past two years, I have worked to transform the course ITIS 3246 (previously ITIS 3110/L) (\emph{IT Infrastructure and Security}) to take advantage of active learning techniques that I have observed and learned about here within the College of Computing and Informatics. The course was originally taught as a traditional lecture/lab style course, and based on feedback from students during my first semester, I converted the course to a flipped classroom. While the initial response and feedback from the students was not positive, I have found that students generally enjoy the course and several have returned in subsequent semesters to tell me how much they learned in the course, and how those skills are now helping them in their jobs and internships.

In the Spring 2018 semester, I taught a ``Topics in SIS'' course titled \emph{Competitive Cyber Defense}. The goal of the course is to help students prepare for competitions that expose them to a hostile environment. In this environment, they are acting as the defenders of an enterprise network that is under active attack. The goal of the competition is to defend the network while also carrying out traditional IT functions, such as installing new software, adding user accounts, and ensuring that services remain online. In Spring 2019, I expanded this course to include a graduate section, and also submitted two Curriculog proposals to make these courses permanent offerings.

Much like ITIS 3246, I have decided to teach \emph{Competitive Cyber Defense} as a flipped class. In Spring 2019, I explored using short lectures that aim to enhance the material studied outside of the classroom. These lectures forgo traditional slides, and instead I use the Socratic method to guide the lecture, with students providing many of the ideas. I have found this to be immensely successful in engaging the students in the discussion, and it is a practice that I plan to continue in all of my classes.
}

\hypertarget{courses}{%
\subsection{Courses Taught}\label{courses}}

\hypertarget{grad-courses}{%
\subsubsection{Graduate Courses}\label{grad-courses}}
\begin{itemize}
\tightlist
\item
  ITIS 6010: \emph{Topics in SIS: Competitive Cyber Defense}: Spring
  2019, Average enrollment: 18 students, Note: Starting Spring 2020 this will be ITIS 5246
\end{itemize}

\hypertarget{undergrad-courses}{%
\subsubsection{Undergraduate Courses}\label{undergrad-courses}}
\begin{itemize}
\tightlist
\item
  ITIS 3110: \emph{IT Infrastructure II: Design and Practice}: Fall
  2017, Spring 2018, Average enrollment: 50 students
\item
  ITIS 3246: \emph{IT Infrastructure and Security}: Fall 2018, Spring
  2019, Fall 2019, Average enrollment: 53 students
\item
  ITIS 4010: \emph{Topics in SIS: Competitive Cyber Defense}: Spring
  2018, Spring 2019, Average enrollment: 16 students, Note: Starting Spring 2020 this will be ITIS 4246
\end{itemize}


\hypertarget{service}{%
\section{Service and Outreach}\label{service}}
\iftoggle{REVIEW}{
  \hypertarget{service-accomplishments}{%
\subsection{Accomplishments}\label{service-accomplishments}}

During my career, I have been fortunate to serve on many program and organizing committees as part of my external service. I have been actively involved in several of the top computer and network security conferences both as a program committee member and as an organizing committee member. As part of being actively involved, I was invited this past year to serve as co-PC chair for the International Workshop on the Theory and Practice of Provenance (TaPP). This workshop is one that is directly relevant to my research, and in 2020, I will serve as the General Chair for Provenance Week, a series of co-located workshops all focused on data provenance.

Within the College, I have taken an active role in serving as a mentor for the 49th Security Division, the ethical hacking and computer security student organization. I have worked closely with the officers of the club to develop changes to their constitution to ensure that all students are welcome within the organization. I have also worked to help the students develop a testbed for the club that is sustainable. This work has resulted in consolidating their infrastructure and teaching them best practices in infrastructure management, concepts which they are first exposed to in some courses, but rarely without a hands-on component that addresses server maintenance and datacenter operations.

Additionally, this year I have started advising undergraduates within the College as part of the Undergraduate Advising Committee. Previously, student advising was handled by the advising center and then by the teaching faculty within the departments. I feel that this role has been valuable in helping to understand the challenges students face and will help shape proposed changes to the curriculum and policies that will be in the best interest of the students.

Finally, one of my goals in the past two years has been to unify the security faculty within the College both professionally and socially. The end result will be a research group that is strong and collaborative, which will help teach our students how to conduct collaborative research. As part of these efforts, I have initiated a ``security reading group'' and a ``security social hour''. Both are open to all that have an interest in security and aim to expose students to new ideas that they otherwise might not be exposed to. These are activities that I will continue to coordinate in the coming years, in order to enhance the collaboration between the faculty and students.

}

\hypertarget{external-service}{%
\subsection{External Service}\label{external-service}}

\hypertarget{talks}{%
\subsubsection{Invited Talks}\label{talks}}

\begin{enumerate}
\tightlist
\item \fullcite{t009}
\item \fullcite{t008}
\item \fullcite{t007}
\item \fullcite{t006}
\item \fullcite{t005}
\item \fullcite{t004}
\item \fullcite{t003}
\item \fullcite{t002}
\item \fullcite{t001}
\end{enumerate}

\hypertarget{journal-reviews}{%
\subsubsection{Journal Reviewer}\label{journal-reviews}}
\small{\textit{Years removed for brevity}}
\begin{itemize}
\tightlist
\item ACM Cloud Computing Security Workshop (CCSW)
\item ACM Computer and Communications Security Conference (CCS)
\item ACM Symposium on Access Control Models and Technologies (SACMAT)
\item ACM Transactions on Internet Technology (TOIT)
\item ACM Transactions on Privacy and Security (TOPS)
\item Annual Computer Security Applications Conference (ACSAC)
\item Future Generation Computer Systems (FGCS)
\item IEEE Embedded Systems Letters (ESL)
\item IEEE International Conference on Computer Communications (INFOCOM)
\item IEEE International Symposium on Hardware Oriented Security and Trust (HOST)
\item IEEE Security and Privacy Magazine (S\&P)
\item IEEE Symposium on Security and Privacy (Oakland)
\item IEEE Transactions on Big Data (TBD)
\item IEEE Transactions on Dependable and Secure Computing (TDSC)
\item IEEE Transactions on Software Engineering (TSE)
\item International Conference on Information Security and Assurance (ISA)
\item International Conference on Information Systems Security (ICISS)
\item Packt Publishing
\item Springer-Verlag Transactions on Computational Science (TCS)
\item USENIX Security Symposium (USENIX Security)
\item USENIX Workshop on Hot Topics in Security (HotSec)
\item Wiley Software Practice and Experience (SPE)
\item Workshop on Virtual Machine Security (VMSec)
\end{itemize}

\hypertarget{organizing-committees}{%
\subsubsection{Organizing Committees}\label{organizing-committees}}
\begin{itemize}
\tightlist
\item
  \emph{2021}: IEEE Secure Development Conference (SecDev, Treasurer),
\item
  \emph{2020}: IEEE Secure Development Conference (SecDev, Treasurer),
  IEEE Symposium on Security and Privacy (Oakland, Student PC Chair),
  Provenance Week (General Chair), Annual Computer Security Applications Conference (ACSAC, Student Conferenceships Chair)
\item
  \emph{2019}: IEEE Symposium on Security and Privacy (Oakland, Student PC Chair),
  USENIX Workshop on the Theory and Practice of Provenance (TaPP, Co-Chair),
  Annual Computer Security Applications Conference (ACSAC, Student Conferenceships Chair)
\item
  \emph{2018}: IEEE Symposium on Security and Privacy (Oakland, Treasurer),
  Annual Computer Security Applications Conference (ACSAC, Poster and WiP Chair)
\item
  \emph{2017}: IEEE Symposium on Security and Privacy (Oakland, Treasurer),
  Annual Computer Security Applications Conference (ACSAC,Poster and WiP Chair)
\item
  \emph{2016}: Annual Computer Security Applications Conference (ACSAC,Poster and WiP Chair)
\item
  \emph{2015}: Annual Computer Security Applications Conference (ACSAC, Poster and WiP Chair)
\end{itemize}

\hypertarget{program-committees}{%
\subsubsection{Program Committees}\label{program-committees}}
\begin{itemize}
\tightlist
\item
  \emph{2021}: IEEE International Conference on Cyber-Security and Resilience (IEEE CSR)
\item
  \emph{2020}: International Conference on Science of Cyber Security (SciSec), 
  EAI International Conference on Security and Privacy in Communication Networks (SecureComm)
\item
  \emph{2019}: International Conference on Science of Cyber Security (SciSec),
  Premier International Conference for Military Communications (MILCOM),
  IEEE Secure Development Conference (SecDev),
\item
  \emph{2018}: Network and Distributed System Security Symposium (NDSS),
  USENIX Security (Security),
  International Workshop on Theory and Practice of Provenance (TaPP),
  International Conference on Science of Cyber Security (SciSec),
  IEEE Secure Development Conference (SecDev),
  Premier International Conference for Military Communications (MILCOM)
\item
  \emph{2017}: International Conference on Availability, Reliability and Security (ARES),
  International Workshop on Theory and Practice of Provenance (TaPP),
  International Symposium on Stabilization, Safety, and Security of Distributed Systems (SSS),
  Premier International Conference for Military Communications (MILCOM),
  IEEE Secure Development Conference (SecDev)
\item
  \emph{2016}: Annual Computer Security Applications Conference (ACSAC),
  International Conference on Availability, Reliability and Security (ARES),
  Premier International Conference for Military Communications (MILCOM)
\item
  \emph{2015}: Annual Computer Security Applications Conference (ACSAC),
  International Conference on Availability, Reliability and Security (ARES),
  Premier International Conference for Military Communications (MILCOM)
\item
  \emph{2014}: Annual Computer Security Applications Conference (ACSAC),
  International Conference on Availability, Reliability and Security (ARES)
\item
  \emph{2013}: Annual Computer Security Applications Conference (ACSAC),
  International Conference on Availability, Reliability and Security (ARES)
\item
  \emph{2012}: Annual Computer Security Applications Conference (ACSAC),
  International Conference on Availability, Reliability and Security (ARES)
\end{itemize}

\hypertarget{professional-affiliations-memberships}{%
\subsubsection{Professional Affiliations/Memberships}\label{professional-affiliations-memberships}}

\begin{itemize}
\tightlist
\item Member, Association for Computing Machinery (ACM)
\item Member, ACM Special Interest Group on Security, Audit and Control (SIGSAC)
\item Member, Institute of Electrical and Electronics Engineers (IEEE)
\item Member, IEEE Computer Society
\item Member, USENIX Association
\end{itemize}

\iftoggle{REVIEW}{
  
\hypertarget{editorial-boards-panels}{%
\subsubsection{Editorial Boards/Panels}\label{editorial-boards-panels}}

\begin{itemize}
\item Review panel, NSF Secure and Trustworthy Cyberspace (SaTC), Small
\end{itemize}

  %\hypertarget{community-service}{%
%\subsubsection{Community Service}\label{community-service}}

\hypertarget{internal-service}{%
\subsection{Internal Service}\label{internal-service}}

\hypertarget{university-committees}{%
\subsubsection{University Committees}\label{university-committees}}
\begin{itemize}
\item UNCC Faculty Legacy Scholarship Committee, Alternate, AY2019-2021
\end{itemize}

\hypertarget{college-committees}{%
\subsubsection{College Committees}\label{college-committees}}
\begin{itemize}
\item College Secretary, AY2020-2021
\item CCI Technology and Infrastructure Committee, SIS Representative, AY2019-2021
\end{itemize}

\hypertarget{dept-committees}{%
\subsubsection{Department Committees}\label{dept-committees}}
\begin{itemize}
\item Research Committee, Member, AY2017-2018
\item Undergraduate Advising Committee, Member, Fall 2018 to current
\item PhD Steering Committee, Member, Fall 2019 to current
\end{itemize}

\hypertarget{thesis-committees}{%
\subsubsection{Ph.D. Disstertation/Master's Thesis/Baccalaureate (Honors) Committees}\label{thesis-committees}}
\begin{itemize}
\item Abduallah Al Farooq, PhD (conferred Summer 2020), Dissertation advisor: Professor Tom Moyer
\item Md Morshed Alam, PhD student, Dissertation advisor: Professor Weichao Wang
\item Islam Obaidat, PhD student, Dissertation advisor: Professor Meera Sridhar
\end{itemize}

%\hypertarget{other-service}{%
%\subsection{Other Service}\label{other-service}}
}
\iftoggle{REVIEW}{
  
\hypertarget{leadership}{%
\section{Leadership}\label{leadership}}

% CReST Lab
During my time at UNCC, I have tried to take on leadership roles in every aspect of my job. As a faculty member, I view one of my primary roles as one where I demonstrate for students what leadership is and how best to lead others. In that regard, I see several leadership roles that I have taken on in my time here.

The first is the formation of the Cyber Resiliency, Security, and Trust (CReST) Lab. I have modeled my research lab in the same way that the Systems and Internet Infrastructure (SIIS) Lab at Penn State was organized during my tenure as a graduate student in the lab. My advisor worked tirelessly, encouraging students to strive for the best, but also to help others. It became a family, with each member of the lab supporting each other. From simple things like reading a paper before a deadline and offering feedback to everyone showing up for major presentations as a show of solidarity and support. I have strived to teach my students the same ideas, and demonstrate for them that we are all working together not only to achieve our academic goals, but to support each other in every way posssible. My students have taken this idea and run with it, organizing social events, and supporting each other whenever possible, even as some of the first students begin to transition into new careers, they remain in contact with their fellow students, offering whatever support they can. My goal is to model for my students a leader that is willing to work hard and support them not just in their academic career, but even as they move on from UNCC.

% 49th Security division
In addition to the CReST Lab, I work closely with students as a faculty advisor for the 49th Security Divison. This is another area where my goal is to model for the students, those qualities that are vital to being a strong leader. I meet reguarly with the student officers of the club to discuss current matters, and offer advice whenever possible on how to achieve their goals. One area where this is starting to show is in the activity surrounding cyber security competitions. The students form teams to participate in competitions, and I work with them to prepare for the competitions both in the classroom and when practicing outside of class. Many of the students express hesitation becuase they have not participated in these competitions before and I work with other students to encourage everyone to participate, regardless of skill level. I find that this has increased involvement in the club activities by many students that otherwise may not have joined and participated.

% Shaping IT within CCI
As I have worked on building the CReST Lab and also supporting the 49th Security Division, I have worked closely with the Technology Solutions Office (TSO) on how research computing equipment is managed within the College. I have worked with several other faculty to interface with both TSO and University IT Services to identify policies and procedures that have delayed research progress. I have worked with these groups to identify a path forward that satisfies the tensions between IT security, University policy, and researchers that have ``non-traditional'' IT needs. Through these efforts, I feel that I have developed a working relationship with TSO and University IT Services that will enable the College to adapt to ever-changing research needs while maintaining a collaborative working environment that satisfies all parties.

% TaPP leadership
Finally, I have been actively involved in the International Workshop on the Theory and Practice of Provenance (TaPP) for several years. I started as an author, submitting my research to the workshop. I then became involved in the program committee for TaPP, becoming a vocal proponent of shaping TaPP as a venue for research in the area of systems provenance. Through my participation on the program committee, I was invited to serve as co-PC chair in 2019, where I was able to advocate more strongly for systems researchers looking for a venue to submit their work. Finally, this coming year (Summer 2020), I have been invited to serve as General Chair for Provenance Week, a series of co-located events that includes TaPP. I have also been invited to sit on the steering committee for TaPP where I will continue to play an active role in shaping the workshop into a venue for systems researchers to publish their work on data provenance.

  \clearpage
\hypertarget{research-statement}{%
\section{Research Statement}\label{research-statement}}


I have spent the majority of my research career focused on problems related to ensuring the integrity of systems and the data they process. This started back in graduate school, and has continued through today. Over the years, the focus has shifted slightly as the problem space changes, and new technologies are developed that address research challenges. The first area that I considered in graduate school examined ways to prove the integrity of a server that is hosting data. Later in my graduate career, the focus shifted to include a rudimentary form of data provenance (history of data), tracking the history of rows in a database. As I transitioned from being a graduate student to a researcher at MIT Lincoln Laboratory, many of these same challenges emerged, and I continued to study the ideas necessary to prove that a system is in a known good state. As I started to examine these challenges in a real-world setting for the Department of Defense, I learned a valuable lesson. Namely, that software is relatively simple to protect, but the data that often drives the most critical decisions is much harder to protect. This has become the central focus of my research over the past several years, and will continue to be an area where I focus my efforts in the future.

My focus on system integrity began with a very simple sounding question. ``How do I know that the web site I connect to that shows me all of the correct security indicators, such as a green padlock icon in the address bar, is running the correct software?'' As it turns out, this is a challenging question, and one that is even more challenging when you have the dual goal of also ensuring that the web server can continue to provide a reasonable throughput. This question began with an exploration of trusted hardware that can be used to ``measure'' the state of a system. Here, the state is constrained to static data and software running on the system. Ultimately, I was able to build a system that could serve thousands of requests per second while providing strong guarantees about the data and software on the system. This system was then expanded to handle dynamic content, and later to include rudimentary provenance, all while maintaining acceptable performance. 

This work transitioned to a real-world deployment for the Department of Defense as I transitioned from a graduate student at Penn State to a research scientist at MIT Lincoln Laboratory. The core ideas of proving system integrity were integrated into a prototype system for the United States Army. As this prototype was being developed, my colleagues and I quickly realized that while the software was static, and thus simpler to validate, the data that drove many of the critical decisions was not static, and was often processed by other systems before being used by our prototype. This led to the natural extension of the questions we had been asking, namely, ``How can we validate the data being used in critical decision processes?''

Ultimately, this led to the use of data provenance within these systems. Initial efforts focused on gathering provenance from and validating the execution simple data processing pipelines. As this line of questioning developed, again the question came up ``How can we prove that this is actually true in the face of an adversary that controls the system?'' To answer that question, a summer intern and I developed a new framework within the operating system that provides strong guarantees about the integrity of the provenance data being collected. As this framework was deployed, it has led to a number of other research challenges and opportunities. Two of the major challenges that I will continue to focus on in the near future are the sheer volume of data being generated\footnote{On the order of one DVD worth of data every 10 minutes when the system is fully loaded.} and the lack of context in the gathered data. As a concrete example, consider a database that stores data in rows and columns. This data has structure that is important in understanding how critical the data is. However, from the operating system's perspective, the database is reading a writing to random blocks on disk, a problem known as a \emph{semantic gap}.

While the problems in data provenance are interesting, they will not have direct real-world impact. Instead, my focus has shifted slightly from ``How can we build trustworthy and efficient provenance systems?'' to ``How can we leverage provenance to build systems that can respond to an attacker without significant human intervention?'' I liken this to the idea of the immune system in the human body. The fundamental question I am trying to answer is ``Now that the system has been compromised, what can be done to evict the attacker, restore the system and data, and evolve to prevent future attacks?'' Building such resilient systems will have tremendous impact as we continue to hear about more and more systems being compromised and sensitive data being leaked at every turn.
  \clearpage
\hypertarget{teaching-statement}{%
\section{Teaching Statement}\label{teaching-statement}}

% Hands on
I am a firm believer that the best way to learn is by doing, and this a how I teach every class that I have taught while at UNC Charlotte. I have been fortunate that the classes I teach have lent themselves to this style of teaching, where students have an opportunity to apply the techniques and skills that I teach. My experiences have led to to further understand the value in having skilled practitioners in the room to assist students. I have also learned through interaction with my peers here in CCI that this is one type of ``active learning'', where I am helping the students not only with the skills being taught in my class, but also teaching them how to become lifelong learners, something that I value greatly in my own life, and am proud to help students on this path of a lifetime of learning.

% Passive consumption is not for the classroom
My education was a very traditional education, with many classes being taught as standard lectures, interspersed with lab assignments when the topic warranted some amount of hands on activity. While I was able to succeed in such an environment, I have first-hand experience with several of my peers as a student who struggled with this style of teaching, often expressing a distaste for lectures, and seeming to only enjoy the labs. This caused me to question if this style of teaching was really most effective, especially in a computer science curriculum where many of the concepts can be read about on-line and then applied in a ``safe environment''. This idea has permeated my own teaching style in such a way that I try to integrate as many hands-on experiences for students as possible, especially in the classroom, where myself and the TA can serve as guides.

When I first considered the flipped classroom approach, where lectures would be presented as videos that students would be expected to watch before class, I asked students what they thought of the idea. Initially students were vocal about how much they did not enjoy this style of teaching. I began to question them, trying to understand the underlying problem, knowing that my own experience of a traditional lecture-based education was not the most ideal method for learning computer science. What I discovered was that students felt that the ``lecture material'' was being outsourced to others, such as Coursera, edX, and other on-line MOOCs, instead of professors recording their own lecture material and then providing short in-class discussions about the lectures. This has stuck with me, and I have ensured that all of my lectures are recorded by me, and fit the topics being covered in the course, only providing links to other videos as additional resources, instead of being the primary resource.

% Socratic lecture style
When I first started recording lectures, it became abundantly clear to me that I still needed to find ways to encourage students to watch the videos before class, enabling them to come to class prepared to apply the concepts they had seen in the videos. I tried a variety of different techniques over the last few semesters with varying degrees of success. At first, I created quizzes that would touch on the topics in the videos, but I found that students could easily search for many of the answers, avoiding having to watch the videos. I found that students would only watch videos in preparation for exams, a technique which has not worked well for them, given the volume of information being covered. Next, I started including questions on the assignments instead of as separate quizzes. These questions required them to think more critically about the content, such that a simple Google search was no longer sufficient to answer the questions. This seemed to be effective for some, but many expressed a dislike for this style of question, feeling that they were often subjective. I have kept these questions on the assignments, but have added some additional discussion time at the beginning of each class to cover the subjective side of these topics.

At the beginning of each class, I know set aside roughly 30-45 minutes (or more if necessary) where I have a Socratic-style lecture/discussion about the topic being covered. I have no slides to present, and instead ask questions and let the students guide the discussion. I keep a list of important points that need covered, but let the students guide the discussion through pointed questions. I have found that students have responded positively to this style of discussion, saying that they feel that the discussion was allowed to cover the topics that students had questions about, and could move quickly through the topics that students demonstrated a firm understanding of by answering my initial questions quickly. The most satisfying experience during this style of discussion was witnessing normally quiet students speaking up to answer another student's question before I had a chance to answer.

My journey as a teacher has only just begun, but I feel that I have grown tremendously in my first two years as an educator. I have moved from a belief that traditional lectures are the correct style of teaching to understanding that there is not a single ``best'' teaching style, and instead, that my role as an educator is to provide students with the tools they need to become lifelong learners.
  
\newpage
\setcounter{page}{1}
\fancyfoot[C]{Peer Institutions}

\begin{quotationb}
Peer institutions with active research in the area(s) of your primary research contributions. The focus here is on research activity at institutions that you consider comparable to UNC Charlotte. The UNC Charlotte Office of Institutional Research maintains a formal list of peer institutions (ir.uncc.edu/peer-institutions), but your response is not limited to that set. If you are not aware of peer institutions with comparable research in your area, please state such.
\end{quotationb}

\begin{enumerate}
\item Florida International University (cloud computing, security of critical infrastructure networks, IoT, smart home security)
\item Old Dominion University (cyber-physical systems)
\item University of Colorado at Denver (proactive security for cyber threats, security for cyber-physical systems \& critical infrastructure, security for Internet-of-Things)
\item University of Nevada at Las Vegas (analytics for security logs)
\item University of New Mexico (system and network security)
\item University of Rhode Island (digital forensics)
\item University of Texas at San Antonio (cloud security, data provenance)
\end{enumerate}

\begin{quotationb}
Peer research groups with active research in the area(s) of your primary research contributions. The focus here is on research activity most closely related to your work across all institutions.
\end{quotationb}

\begin{enumerate}
\item University of Texas at San Antonio (cloud security, data provenance)
\item University of Illinois at Urbana-Champaign (data provenance, systems security)
\item University of Florida (data provennce, cloud security, system security)
\item The Pennsylvania State University (system security)
\item NC State University (system security)
\item University of British Columbia (systems research, data provenance)
\item University of Bristol (data provenance)
\item University of Cambridge (data provenance, systems security)
\item Stony Brook University (data provenance)
\end{enumerate}

  \newpage
\setcounter{page}{1}
\fancyfoot[C]{Publication Venues}

\section*{Journals}
Impact factors come from the inCites database available through Atkins Library. Journal h-index values come from \url{https://www.scimagojr.com/}.
\begin{enumerate}
\tightlist
\item \textbf{IEEE Transactions on Computers}

Top-tier journal for a broad range of computer science topics, including systems and security research.

[ impact factor: 3.131, h-index: 110 ]

\item \textbf{ACM Transactions on Internet Technology (TOIT)}

Journal that focuses on all areas of network and web systems, digital public policy, and other technically oriented issues on the design, use, and services of the Internet.

[ impact factor: 2.382, h-index: 46 ]

\end{enumerate}

\section*{Conferences[Security]}
All of the conferences in this section are top-tier security conferences with the exception of ACSAC. Acceptance rates come from \url{http://faculty.cs.tamu.edu/guofei/sec_conf_stat.htm} unless otherwise noted.

\begin{enumerate}[resume]
\tightlist
\item \textbf{IEEE Symposium on Security and Privacy (S\&P)}

[ Acceptance Rate (2019): 12\% ]

\item \textbf{Network and Distributed System Security Symposium (NDSS)}

[ Acceptance Rate (2019): 17\% ]

\item \textbf{ACM Conference on Computer and Communications Security (CCS)}

[ Acceptance Rate (2018): 16.6\% ]

\item \textbf{USENIX Security}

[ Acceptance Rate (2018): 19.1\% ]

\item \textbf{Annual Computer Security Applications Conference}

ACSAC is considered a tier-2 conference that focuses on applied security. Many government agencies send representatives to this conference, making it an ideal place to network with potential sponsors in the US government, especially the Departments of Homeland Security and Defense (DHS and DoD).

[ Acceptance Rate (2017): 19.7\% ]

\end{enumerate}

\section*{Conferences[Systems]}
\emph{Note:} OSDI and SOSP alternate years with SOSP running in odd-numbered years and OSDI running in even-numbered years. The list of topics are similar.
\begin{enumerate}[resume]
\tightlist
\item \textbf{ACM Symposium on Operating Systems Principles (SOSP)}

\textbf{Theme/Topics:} \emph{(From the 2019 CFP)} ``SOSP takes a broad view of the systems area and solicits contributions from many fields of systems practice, including, but not limited to, operating systems, file and storage systems, distributed systems, cloud systems, mobile systems, secure systems, embedded systems, dependable systems, system management and virtualization. We also welcome work that explores the interface to related areas such as computer architecture, networking, programming languages, and databases.''

[ Acceptance Rate (2017): 16.8\% ]

(\url{https://dl.acm.org/citation.cfm?id=3132747})

\item \textbf{USENIX Symposium on Operating Systems Design and Implementation (OSDI)}

\textbf{Theme/Topics:} \emph{(From the 2018 CFP)} ``OSDI takes a broad view of the systems area and solicits contributions from many fields of systems practice, including, but not limited to, operating systems, file and storage systems, distributed systems, cloud computing, mobile systems, secure and reliable systems, systems aspects of big data, embedded systems, virtualization, networking as it relates to operating systems, and management and troubleshooting of complex systems. We also welcome work that explores the interface to related areas such as computer architecture, networking, programming languages, analytics, and databases.''

[ Acceptance Rate (2018): 18.3\% ]

(\url{https://www.usenix.org/sites/default/files/osdi18_message.pdf})

\item \textbf{USENIX Annual Technical Conference (ATC)}

\textbf{Theme/Topics:} The scope of ATC is quite broad and includes architectural interaction, big data infrastructure, cloud and edge computing, distributed and parallel systems, embedded systems, energy/power management, file and storage systems, Internet of THings, machine learning and systems interactions, mobile and wireless, networking (WAN, LAN, datacenter) and network services, operating systems, reliability, availability, and scalability, security, privacy, and trust, system and network management and troubleshooting, usage studies and workload characterization, and virtualization.

[ Acceptance Rate (2018): 20.2\% ]

(\url{https://www.usenix.org/sites/default/files/atc18_message.pdf})

\end{enumerate}

\section*{Workshops}
\begin{enumerate}[resume]
\tightlist
\item {International Workshop on the Theory and Practice of Provenance (TaPP)}

\textbf{Theme/Topics:} TaPP is a workshop that focuses on data provenance including the following topics: provenance management system prototypes and commercial solutions, provenance analytics, querying, and reasoning about provenance, visualizing provenance information, performance aspects of provenance capture, storage, and analytics, standardization of provenance models and representations, security and privacy implications of provenance, applications of provenance in real life settings, human interaction with provenance, retroactive reconstruction of provenance, using provenance for evaluating data quality and trust in data, novel methods for capturing provenance, integrating provenance information, interoperability among provenance-aware systems, and provenance discovery.

\textbf{Note:} In odd-numbered years TaPP is co-located with relevant conferences. In the even-numbered years TaPP is part of Provenance Week, an event which includes TaPP, the International Provenance and Annotation Workshop (IPAW), and several smaller workshops.

[ Acceptance Rate (2019): 66.7\% (6/9) ]

\end{enumerate}
}

\end{document}
